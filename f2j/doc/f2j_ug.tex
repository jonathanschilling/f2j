\documentclass[11pt]{article}
\setlength{\oddsidemargin}{.25in}
\setlength{\topmargin}{-.25in}
\setlength{\textheight}{8.75in}
\setlength{\textwidth}{6in}
\setlength{\parindent}{.25in}
\usepackage{moreverb}
\usepackage{longtable}
\usepackage{textcomp}
\usepackage{graphicx} 
\usepackage{amstext,amssymb}
\usepackage{pslatex}
\usepackage{url}
\usepackage[ps2pdf,colorlinks]{hyperref}

\begin{document}

\pagenumbering{roman}
\thispagestyle{empty}
\begin{center}
\huge
\bf
User's Guide to f2j \\
Version 0.8
\vspace*{1in} \mbox{} \\
\LARGE \rm
Keith Seymour and Jack Dongarra 

\vspace*{.5in}
Innovative Computing Laboratory\\
Department of Computer Science\\
University of Tennessee\\
\vspace*{.5in}
May 31, 2007
\end{center}


%\include{license}
%\tableofcontents
%\listoftables
%\listoffigures

\newpage
\pagenumbering{arabic}
\setcounter{page}{1}

\section{Introduction}

Before using the f2j source code, realize that f2j was originally geared
to a very specific problem - that is, translating the LAPACK and BLAS numerical
libraries.  However, now that the translation of LAPACK is complete, the goal is
to handle as much Fortran as possible, but there's still a lot left to cover.
We have a lot of confidence in the JLAPACK translation, but for a variety of
reasons, f2j will most likely not correctly translate your code at first.

One of the reasons for putting the code up on SourceForge is to enable
collaboration with other developers.  If you're interested in helping the
development of f2j, we'll be glad to give commit access to the CVS tree.

The purpose of this document is to describe how to build and use the f2j compiler and
to give some background on how to extend it to handle your Fortran code. 

\section{Obtaining the Code}

For downloads and CVS access, see the f2j project page at SourceForge:

  http://sourceforge.net/projects/f2j

The GOTO translation code is based on the bytecode parser found in javab, a
bytecode parallelizing tool under development at the University of Indiana.
This code is covered under its original license, found in the translator source
directory.

\section{Limitations}

Complex data types are still not supported.  f2j also doesn't support file
I/O, certain forms of EQUIVALENCE, and has pretty weak formatted I/O
support.  There are a couple of parsing quirks too.

With that said, if you have pretty straightforward numerical code (similar to
BLAS or LAPACK) f2j may be able to handle it.

\section{Building and Using f2java}

We have been doing development and testing of f2j on Sun SPARCstations running
various versions of Solaris as well as x86 machines running various versions
of Linux and Solaris/x86.  It may compile on
other platforms, though.  At least it's worth a shot.  Depending on the compiler,
you may see some warnings about unused variables, etc.  You can safely ignore
them.

First, download and uncompress the source code.  Building the code follows the typical configure/make process:

\begin{verbatim}
# ./configure
# make
\end{verbatim}

Optionally, you can ``make install'' which will copy the executables to the
location specified in the \verb|--prefix| argument to configure.

Now you may want to add the relevant install directory to your PATH.  This
will vary depending on whether you did ``make install''.  If so, the PATH
should include {\verb $prefix/bin}.  If not, your PATH should include
\verb|$f2j_dir/src| and \verb|$f2j_dir/goto_trans|, where \verb|$f2j_dir|
is the top-level f2j source directory.

Let's go through a simple example.  Say you have the following Fortran code in a file 
called ``test.f''

\begin{verbatim}
      program blah
      external foo
      write(*,*) 'hi'
      call foo(12)
      stop
      end
      subroutine foo(y)
      integer y
      write(*,*) 'foo ', y
      return
      end
\end{verbatim}

If you translate it with ``\verb|f2java test.f|'', it will produce one class file and
one Java source file for each program unit.  So, in this case since we have 
two program units in the Fortran source file, we end up with four generated files:
Blah.java, Blah.class, Foo.java, and Foo.class (note the first letter
of the name becomes
capitalized).  You can run the generated class file directly:

\begin{verbatim}
# java Blah
hi
foo  12
\end{verbatim}

You don't need to compile the Java source, but if you wanted to modify it,
you could recompile:

\begin{verbatim}
# javac Blah.java Foo.java
\end{verbatim}

However at this point the GOTO statements haven't been converted, so if you
run it you'll see some warnings like this:

\begin{verbatim}
# java Blah
hi
foo  12
Warning: Untransformed goto remaining in program! (Foo, 999999)
Warning: Untransformed label remaining in program! (Foo, 999999)
\end{verbatim}

So you need to run the GOTO transformer (javab) on the class files:

\begin{verbatim}
# javab *.class
\end{verbatim}

and then it'll run fine:

\begin{verbatim}
# java Blah
hi
foo  12
\end{verbatim}

\section{Command-line Options}

There are several command-line options that you should be aware of: 

\begin{itemize}
\item -I specifies a path to be searched for included files (may be used
  multiple times).

\item -c specifies the search path for ``f2j'' files.
  It is a colon-separated list of paths, like a Java CLASSPATH).  For example:
\begin{verbatim}
    f2java -c .:../objects filename.f
\end{verbatim}

\item -p specifies the name of the package.  For example:
\begin{verbatim}
    f2java -p org.netlib.blas filename.f
\end{verbatim}

\item -o specifies the destination directory to which the code should be
  written.

\item -w forces all scalars to be generated as wrapped objects.  The default
  behavior is to only wrap those scalars that must be passed by reference.
  Note that using this option will generate less efficient Java code.

\item -i causes f2j to generate a high-level interface to each subroutine and
  function.  The high-level interface uses a Java-style calling convention (2D
  row-major arrays, etc).  The low-level routine is still generated because the
  high-level interface simply performs some conversions and then calls the
  low-level routine.

\item -h displays help information.

\item -s causes f2j to simplify the interfaces by removing the offset parameter and
  using a zero offset.  It isn't necessary to specify -i in addition to -s.

\item -d causes f2j to generate comments in a format suitable for javadoc.  It is a
  bit of a LAPACK-specfic hack -- the longest comment in the program unit is
  placed in the javadoc comment.  It works fine for BLAS/LAPACK code (or any
  other code where the longest comment is the one that describes the function),
  but will most likely not work for other code.

\item -fm causes f2j to generate code that calls java.lang.StrictMath
  instead of java.lang.Math.  By default, java.lang.Math is used.

\item -fs causes f2j to declare the generated code as strictfp (strict
  floating point).  By default, the generated code is not strict.

\item -fb enables both the -fm and -fs options.

\item -vs causes f2j to generate all variables as static class
  variables.  By default f2j generates variables as locals.

\item -va causes f2j to generate arrays as static class variables, but
  other variables are generated as locals.
\end{itemize}

After issuing the command ``f2java file.f'' there should be one or more Java
files in your current directory, one Java file and one class file per Fortran program unit
(function, subroutine, program) in the source file.  Initially, we would
suggest concatenating all Fortran program units into one file because it makes
it easier to perform correct code generation (more about this later).  As the
example above illustrated, you can run the class file corresponding to the main
Fortran program unit or you can use the Java compiler of your choice to compile
the resulting Java source code.  Make sure that the org.netlib.util package is
in your CLASSPATH.  This package comes in the JLAPACK distribution, so if your
CLASSPATH already points to JLAPACK's f2jutil.jar, then you're ok.

\section{Extending f2j}

So, at this point you may be wondering how to extend f2j to handle your code.
Typically, the first problem you'll run into is that f2j doesn't parse your
code. That could involve something as simple as changing a production in the
parser or it could involve a bit more work - e.g. creating a new kind of AST
node along with all the appropriate code generation routines. The first thing
you'll want to check is whether the parser supports the syntax your code uses
(the parsing code is machine generated from a Yacc parser in f2jparse.y). For
example, if your code contains an IMPLICIT statement (besides IMPLICIT NONE),
your code will not compile because f2j doesn't support implicit typing. Suppose
you wanted to implement implicit typing in f2j. Your first step would be to
modify the ``Implicit'' production in the parser to allow the statement. Then
you'd probably have to modify the ``type assignment'' code in typecheck.c, but
you shouldn't need to change the code generator (codegen.c). The second thing
you'll want to check if your code doesn't parse is whether the lexer is acting
up. If you're getting parse errors on a line of code that should compile based
on your examination of the parser, then the lexer might not be sending the
correct tokens to the parser. The lexical analysis code is in f2jlex.c, which
is handwritten C code based on Sale's algorithm. There's not really an easy way
of describing the structure of the lexer code, but if you enable debugging
output (set lexdebug = TRUE) it should help you figure out where the problem
is.

If f2j can parse your code, but the resulting Java code does not compile or
does not work, then this may indicate a problem in the f2j back-end. First, try
concatenating all your Fortran files into one big file (ok, we admit this is
cheesy, but we haven't implemented multiple file handling yet). This should
help with the type analysis phase and may eliminate problems in the resulting
Java code. After that, if the generated code is still incorrect, begin looking
into the f2j code. After f2j parses your code, it passes through a couple of
stages before actually generating code. First, the AST goes through 'type
analysis' (typecheck.c), which simply means that the tree is fully traversed
and each node is assigned type information as appropriate. This is not semantic
analysis, just annotation. Next, the AST goes through 'scalar optimization'
(optimize.c), which is an optimization stage designed to determine which scalar
variables need to be wrapped in objects and which can remain primitives. After
that, f2j generates the Java code (codegen.c) based on the modified AST. So, if
you notice a type mismatch problem in the generated code, typecheck.c would be
a good place to begin debugging. Similarly, if you notice that object wrappers
are inappropriately used, check into optimize.c (hint: by passing the -w flag
to f2java, the scalar optimization code will be skipped). Most other problems
will be with the code generator itself.

If you're ambitious, you might try grafting a full Fortran 77 front-end onto
f2j and start building up the back-end to accommodate all the new AST nodes. Or
you might just try fully implementing Fortran's I/O routines for f2j.  Tips:

\begin{itemize}
\item f2java -h provides a help screen.
\item For best results, concatenate all Fortran files into one source file.
\item Remember to run the goto translation program on the resulting bytecode.
\item Jasmin code generation and VCG graph generation are deprecated.
\item Last line of input can't begin with a TAB.
\end{itemize}

\end{document}
